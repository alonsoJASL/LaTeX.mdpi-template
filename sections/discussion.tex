\vspace{-1em}
\section{DISCUSSION}\label{sec:discussion}
\vspace{-0.25em}
Preliminary work involving curve evolution methods such as multilevel set
methods \cite{lu2015} and active contours \cite{chanvese} have proven unsuccessful
to segment even non overlapping cells in the analysis of a single frame.
Such techniques require considerable precision when selecting the parameters,
especially when selecting the initialisation given to the algorithms.
In this work, a method to segment moving cells
%based on active contours
was proposed.

The main contribution was to provide a framework for the consistent tracking of
the shape of a cell and provide a measured evolution of some shape parameters.
As shown in Figure \ref{fig:res-all}, the similarity of the previous frame to the
current is enough to provide an appropriate initialisation for the active
contours, however, as shown in Table \ref{tab:acparams}, monitoring of the shape
is necessary to avoid segmentation leaking.
A new implementation of the \emph{anglegram} matrix allowed for the analysis
of a single cell with a straightforward identification of corners in the shapes.
Future developments involve extending the shape tracking into overlapping cells
to disambiguate them.
Furthermore, as seen in Figure \ref{fig:agshapes}, the distinctive patterns of
the anglegrams corresponding to the basic shapes, could be further explored
into a classification problem for shape identification or correction of
irregular segmentations.
\vspace{-0.5em}
